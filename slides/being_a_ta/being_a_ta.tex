\documentclass{beamer}
\usetheme{metropolis}           % Use metropolis theme
\setbeamercovered{transparent}

\title{Being a TA}
\date{\today}
\author{Mine \c{C}etinkaya-Rundel}
\institute{Sta 771S - Teaching Statistics}

\begin{document}
\maketitle
 
 
% -------------------------------------------------------------------

\section{Overview}

% -------------------------------------------------------------------

\begin{frame}
\frametitle{What is a TA?}

What are (and are not) the responsibilities of a TA?

\vfill

\end{frame}

% -------------------------------------------------------------------

\section{Running labs}

% -------------------------------------------------------------------

\begin{frame}
\frametitle{Introductions}

\begin{itemize}

\item Introducing yourself

\item Introduction of each student -- worthwhile if class is small

\end{itemize}

\end{frame}

% -------------------------------------------------------------------

\begin{frame}
\frametitle{Getting and retaining student attention}

\begin{itemize}[<+->]

\item You win or lose the attention of the students in the first 5 minutes

\item The most effective way to get their attention (and respect) is to be (and appear as) an authority on the content

\item The most effective way to lost these is to assume you can just wing it (especially when ``it" is something as structured as a computational lab)

\item Other tips: Eye contact, walking around, getting buy in by involving them in the introduction of the material

\end{itemize}

Example: \href{https://www2.stat.duke.edu/courses/Spring16/sta101.001/post/labs/intro_to_r.html}{\textcolor{orange}{STA 101 - Introduction to R and RStudio}}

\end{frame}

% -------------------------------------------------------------------

\begin{frame}
\frametitle{Guidelines: running labs}

\begin{itemize}[<+->]

\item Be clear on instructor expectations from you and the students

\item Work through the lab on the same platform that the students will be using

\item Be familiar with syntax course uses

\item Don't be afraid of live debugging

\item Do not debug student code directly, help them figure out the issue

\end{itemize}


\end{frame}

% -------------------------------------------------------------------

\begin{frame}
\frametitle{Sticky situations...}

Imagine the following scneario:

\begin{itemize}

\item Student A is very confused, and keeps interrupting you to ask questions.

\item Student B and C get annoyed with Student A's questions, they start rolling their eyes and quietly (but audibly) making judgmental remarks about Student A.

\end{itemize}

How do you handle this situation?

\end{frame}

% -------------------------------------------------------------------

\section{Office hours}

% -------------------------------------------------------------------

\begin{frame}
\frametitle{Mock office hours - 10 minutes}

One of you is the student and the other is the TA. Student asks the following question: 

\vfill

\textit{Why is it true that in simple linear regression $b_0 = \bar{y} - b_1 \bar{x}$?}

\vfill

Assume that this is part of the assignment student needs to turn in, i.e. you cannot just give away the answer. Instead you need to get the student to understand the answer well enough to be able to craft a response in their own words.

\vfill

TA: Explain, Student: Write out a brief answer.

\end{frame}

% -------------------------------------------------------------------

\begin{frame}
\frametitle{Guidelines: office hours}

\begin{itemize}[<+->]

\item Be familiar with text and class notes: use same notation, definition, perspectives as author of text and instructor of course

\item Show interest in helping and being (and appearing) welcoming

\item Avoid solving HW problems for students: ask guiding, probing questions, but get the student to do the thinking

\item Work from the level of the student: which requires that you know (can evaluate) the level of the student

\item Speak clearly, slowly, and audibly

\item Do not fake it: if you do not know the answer, seek guidance, do not give incorrect information

\end{itemize}

\end{frame}

% -------------------------------------------------------------------

\section{Answering student questions online}

% -------------------------------------------------------------------

\begin{frame}
\frametitle{Guidelines: Email}

\begin{itemize}[<+->]

\item Use your professional (Duke) email address if you can

\item Read one more time before you hit send

\item Do not invite a back-and-forth conversation

\end{itemize}

\end{frame}

% -------------------------------------------------------------------

\begin{frame}
\frametitle{Guidelines: Piazza}

\begin{itemize}[<+->]

\item Use the setting for ``students can opt to be anonymous to other students, but not to instructors"

\item Be careful about not using the student's name if they posted anonymously

\end{itemize}

\vfill

How should (if at all) your responses on Piazza differ from email?

\vfill

\end{frame}

% -------------------------------------------------------------------

\section{Grading}

% -------------------------------------------------------------------

\begin{frame}
\frametitle{Mock grading - 5 minutes}

\begin{itemize}

\item Mark where you take of points

\item Provide constructive feedback, but be realistic about how much you write (imagine having to do this for many students in a timely manner)

\item Make sure your writing is legible

\end{itemize}

\end{frame}

% -------------------------------------------------------------------

\begin{frame}
\frametitle{Sticky situations...}

Suppose a students asks \textit{``I lost 10 points on this question, but my friend who has the same answer only lost 5 points"}.

How do you handle this situation?

\end{frame}

% -------------------------------------------------------------------

\section{Visit a lab}

% -------------------------------------------------------------------

\begin{frame}
\frametitle{Visit a lab}

Take a look at this semester's course, specifically lab, schedule.

Pick a lab that you would like to visit.

Post on Piazza by Friday which lab you will be visiting.

I will post a form on Piazza for you to fill out with comments.

One page lab visit reflections \textbf{due March 1}.

\end{frame}

% -------------------------------------------------------------------

\end{document}